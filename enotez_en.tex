% arara: pdflatex
% arara: biber
% arara: pdflatex
% arara: pdflatex
% --------------------------------------------------------------------------
% the ENOTEZ package
% 
%   Endnotes for LaTeX2e
% 
% --------------------------------------------------------------------------
% Clemens Niederberger
% Web:    https://bitbucket.org/cgnieder/enotez/
% E-Mail: contact@mychemistry.eu
% --------------------------------------------------------------------------
% Copyright 2011-2012 Clemens Niederberger
% 
% This work may be distributed and/or modified under the
% conditions of the LaTeX Project Public License, either version 1.3
% of this license or (at your option) any later version.
% The latest version of this license is in
%   http://www.latex-project.org/lppl.txt
% and version 1.3 or later is part of all distributions of LaTeX
% version 2005/12/01 or later.
% 
% This work has the LPPL maintenance status `maintained'.
% 
% The Current Maintainer of this work is Clemens Niederberger.
% --------------------------------------------------------------------------
% The enotez package consists of the files
%  - enotez.sty, enotez_en.tex, enotez_en.pdf, README
% --------------------------------------------------------------------------
% If you have any ideas, questions, suggestions or bugs to report, please
% feel free to contact me.
% --------------------------------------------------------------------------
\documentclass[toc=bib]{cnpkgdoc}
\docsetup{
  pkg      = enotez ,
  code-box = {
    backgroundcolor  = gray!7!white ,
    skipbelow        = .6\baselineskip plus .5ex minus .5ex ,
    skipabove        = .6\baselineskip plus .5ex minus .5ex ,
    roundcorner      = 3pt ,
  } ,
  gobble   = 1 ,
  subtitle = {Endnotes for \LaTeXe}
}

\addcmds{
  chapter,
  cmd,
  DeclareInstance,
  DeclareTemplateInterface,
  endnote,
  endnotemark,
  endnotetext,
  enmark,
  phantomsection,
  printendnotes,
  textsuperscript
}

\DeclareInstance{enotez-list}{addsec}{paragraph}{heading=\addsec{#1}}

% Layout:
\usepackage[osf]{libertine}
\cnpkgcolors{
  main   => cnpkgred ,
  key    => yellow!40!brown ,
  module => cnpkgblue ,
  link   => black!90
}
\renewcommand*\othersectionlevelsformat[3]{%
  \textcolor{main}{#3\autodot}\enskip}
\renewcommand*\partformat{%
  \textcolor{main}{\partname~\thepart\autodot}}
\usepackage{fnpct}
\AdaptNote\endnote\multendnote
\usepackage{embrac}[2012/06/29]
\ChangeEmph{[}[,.02em]{]}[.055em,-.08em]
\ChangeEmph{(}[-.01em,.04em]{)}[.04em,-.05em]

\ExplSyntaxOn
\NewDocumentCommand \Default {g}
  {
    \hfill\llap
      {
        \IfNoValueTF { #1 }
          {(initially~empty)}
          {Default:~\code{#1}}
      }
    \newline
  }
\ExplSyntaxOff

\usepackage[backend=biber,style=alphabetic]{biblatex}
\addbibresource{\jobname.bib}

% rudimentary solution for a `maintainer' field:
\DeclareFieldFormat{authortype}{\mkbibparens{#1}}
\renewbibmacro*{author}{%
  \ifboolexpr{
    test \ifuseauthor
    and
    not test {\ifnameundef{author}}
  }
    {\printnames{author}%
     \iffieldundef{authortype}
       {}
       {\setunit{\space}%
	\usebibmacro{authorstrg}}}
    {}}

\usepackage{filecontents}
\begin{filecontents*}{\jobname.bib}
@software{endnotes,
  title      = {endnotes},
  author     = {Robin Fairbairns},
  authortype = {current maintainer},
  date       = {2003-01-15},
  version    = {NA},
  url        = {http://www.ctan.org/pkg/endnotes},
  urldate    = {2012-07-03}
}
\end{filecontents*}


\begin{document}

\section{Licence and Requirements}
Permission is granted to copy, distribute and/or modify this software under the
terms of the LaTeX Project Public License, version 1.3 or later
(\url{http://www.latex-project.org/lppl.txt}). The package has the status
``maintained.''

\enotez loads and needs the following packages: \paket{expl3},
\paket{xparse}, \paket{xtemplate} and \paket{l3keys2e}.

\section{Motivation}
\enotez is a new implementation of endnotes for \LaTeXe\ since the \paket{endnotes}
package~\cite{endnotes} has some deficiencies. Nested endnotes, for example, are
not supported, neither is \paket{hyperref}.

\enotez enables nested endnotes properly and has another mechanism of customizing
the list of endnotes which is easily extendable.

\section{Usage}
\subsection{Placing the Notes}
The usage is simple: use \cmd{endnote} in the text where you want to place the
note mark.
\begin{beschreibung}
 \befehl{endnote}[<mark>]{<text>} Add an endnote in the text.
\end{beschreibung}
\begin{beispiel}
 This is some text\endnote{With an endnote.}.
\end{beispiel}
There's not really much more to it. It is possible to add a custom mark by
using the optional argument but that should be needed too often. \cmd{endnote}
works fine inside tables, minipages, floats\ldots. Endnotes can also be nested.
\begin{beispiel}
 This is some text\endnote{With another endnote\endnote{This is a
 nested endnote!}.}.
\end{beispiel}

\subsection{Printing the Notes}
The notes are printed by using the command \cmd{printendnotes}.
\begin{beschreibung}
 \befehl{printendnotes}*[<style>] Print the list of endnotes. \code{<style>} is
   one of the instances explained in section~\ref{ssec:customizing_the_list}.
\end{beschreibung}
If used without argument it prints all notes set so far with \cmd{endnote}. The
current list will then be cleared. All endnotes set after it are stored again
for the next usage of \cmd{printendnotes}. The starred version will print
\emph{all} endnotes but shouldn't be used more than once if you have nested
endnotes.

It may take several compilation runs until all notes are printed correctly. In
a first run they are written to the \code{aux} file. In the second run they are
available to \cmd{printendnotes}. If you have nested endnotes they will be written
to the \code{aux} file the first time they're printed with \cmd{printendnotes}
which means you might have to compile your file once more. If you change any of
the endnotes or add another one you again will need at least two runs. \enotez
tries to warn you in these cases by invoking \LaTeX's warning

\code{Label(s) may have changed. Rerun to get cross-references right.}

\noindent but may not catch all cases.

\section{Options}
\subsection{Package Options}
\enotez has a few package options. They should be pretty self-explanatory.
\begin{beschreibung}
 \Option{list-name}{<list name>}\Default{Notes}
   The name of the notes list. This name is used for the heading of the list.
 \Option{reset}{\default{true}|false}\Default{false}
   If set to \code{true} the notes numbers will start from 1 again after
   \cmd{printendnotes} has been invoked.
 \Option{counter-format}{arabic|alph|Alph|roman|Roman}\Default{arabic}
   Change the format of the endnote counter.
 \Option{totoc}{section|chapter|false}\Default{false}
   Add an entry to the table of contents.
 \Option{list-style}{<style>}\Default{plain}
   Sets the default list style, see section \ref{ssec:customizing_the_list} for
   details.
\end{beschreibung}

\subsection{Customizing the List}\label{ssec:customizing_the_list}
The list is typeset with \paket{xtemplate}'s template mechanism. \enotez declares
the object \code{enotez-list} and two templates for it, the template \code{paragraph}
and the template \code{list}.

\subsubsection{The \code{paragraph} Template}
The \code{paragraph} template's interface is defined as follows:
\begin{beispiel}[code only]
 \DeclareTemplateInterface{enotez-list}{paragraph}{1}
   {
     heading       : function 1 = \section*{#1}   ,
     format        : tokenlist  = \footnotesize   ,
     number        : function 1 = \enmark{#1}     ,
     number-format : tokenlist  = \normalfont     ,
     notes-sep     : length     = .5\baselineskip ,
   }
\end{beispiel}
The parameters functions are these:
\begin{description}
 \item[\code{heading}] The command with which the heading is typeset.
 \item[\code{format}] The format of the whole list.
 \item[\code{number}] The command that is used to typeset the numbers of the
   notes. The command \cmd{enmark} is explained soon.
 \item[\code{numbers-format}] The format of the numbers.
 \item[\code{notes-sep}] Additional space between the notes.
\end{description}

\enotez uses this template to define the instance \code{plain}:
\begin{beispiel}[code only]
 \DeclareInstance{enotez-list}{plain}{paragraph}{}
\end{beispiel}
This is the default style of the list.

You can easily define your own instances, though:
\begin{beispiel}[code only]
 \DeclareInstance{enotez-list}{custom}{paragraph}
   {
     heading   = \chapter*{#1}        ,
     notes-sep = \baselineskip        ,
     format    = \normalfont          ,
     number    = \textsuperscript{#1}
   }
\end{beispiel}
This would use a chapter heading for the title, separate the notes with
\verb=\baselineskip= and typeset them with \verb=\normalfont=. The numbers would
be typeset with \verb=\textsuperscript=. You could now use it like this:
\begin{beispiel}[code only]
 \printendnotes[custom]
\end{beispiel}

If you wanted superscripted numbers, you could also redefine \cmd{enmark}. 
\begin{beschreibung}
 \befehl{enmark} is defined like this: \verb=\newcommand*\enmark[1]{#1.}=
\end{beschreibung}

\subsubsection{The \code{list} Template}
The \code{paragraph} template's interface is defined as follows:
\begin{beispiel}[code only]
 \DeclareTemplateInterface{enotez-list}{list}{1}
   {
     heading       : function 1 = \section*{#1} ,
     format        : tokenlist  = \footnotesize ,
     number        : function 1 = \enmark{#1}   ,
     number-format : tokenlist  = \normalfont   ,
     list-type     : tokenlist  = description   ,
   }
\end{beispiel}
This template uses a list to typeset the notes. As you can see the default list
is a \code{description} list.

\enotez defines two instances of this template:
\begin{beispiel}[code only]
 \DeclareInstance{enotez-list}{description}{list}{}
 \DeclareInstance{enotez-list}{itemize}{list}{list-type = itemize }
\end{beispiel}
They're available through \cmd{printendnotes}[description] and
\cmd{printendnotes}[itemize], respectively.

Again you can define your own instances using whatever list you want, possibly
one defined with the power \paket{enumitem}.

\section{hyperref Support}
If \paket{hyperref} is loaded and you are using the option \key{totoc} (see
p~\ref{key:totoc}) the list title is linked via a \verb=\phantomsection=.

If \paket{hyperref} is used with \code{hyperfootnotes} set to \code{true} the
endnote marks are linked to the respective entries in the list.

\printendnotes[addsec]

{\EmbracOff\printbibliography}

\end{document}