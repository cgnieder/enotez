% arara: pdflatex
% arara: biber
% arara: pdflatex
% arara: pdflatex
% --------------------------------------------------------------------------
% the ENOTEZ package
% 
%   Endnotes for LaTeX2e
% 
% --------------------------------------------------------------------------
% Clemens Niederberger
% Web:    https://bitbucket.org/cgnieder/enotez/
% E-Mail: contact@mychemistry.eu
% --------------------------------------------------------------------------
% Copyright 2011-2013 Clemens Niederberger
% 
% This work may be distributed and/or modified under the
% conditions of the LaTeX Project Public License, either version 1.3
% of this license or (at your option) any later version.
% The latest version of this license is in
%   http://www.latex-project.org/lppl.txt
% and version 1.3 or later is part of all distributions of LaTeX
% version 2005/12/01 or later.
% 
% This work has the LPPL maintenance status `maintained'.
% 
% The Current Maintainer of this work is Clemens Niederberger.
% --------------------------------------------------------------------------
% The enotez package consists of the files
%  - enotez.sty, enotez_en.tex, enotez_en.pdf, README
% --------------------------------------------------------------------------
% If you have any ideas, questions, suggestions or bugs to report, please
% feel free to contact me.
% --------------------------------------------------------------------------
\documentclass[toc=bib,toc=index]{cnpkgdoc}
\docsetup{
  pkg      = enotez ,
  code-box = {
    backgroundcolor  = gray!3!white ,
    skipbelow        = .6\baselineskip plus .5ex minus .5ex ,
    skipabove        = .6\baselineskip plus .5ex minus .5ex ,
    roundcorner      = 3pt ,
  } ,
  gobble   = 1 ,
  subtitle = {Endnotes for \LaTeXe}
}

\addcmds{
  @endnotemark,
  @ifnextchar,
  appendix,
  chapter,
  cmd,
  DeclareInstance,
  DeclareTemplateInterface,
  DeclareTranslation,
  endnote,
  endnotemark,
  endnotetext,
  enmark,
  enmarkstyle,
  enotezwritemark,
  kant,
  phantomsection,
  printendnotes,
  setenotez,
  splitendnotes,
  textsuperscript,
  theendnote
}
\setenotez{mark-cs=\textsu,backref}
\DeclareInstance{enotez-list}{addsec}{paragraph}{heading=\addsec{#1}}

% Layout:
\usepackage{libertinehologopatch}

\cnpkgusecolorscheme{friendly}
\renewcommand*\othersectionlevelsformat[3]{%
  \textcolor{main}{#3\autodot}\enskip}
\renewcommand*\partformat{%
  \textcolor{main}{\partname~\thepart\autodot}}
\usepackage{fnpct}
\AdaptNote\endnote\multendnote
\usepackage{embrac}[2012/06/29]
\ChangeEmph{[}[,.02em]{]}[.055em,-.08em]
\ChangeEmph{(}[-.01em,.04em]{)}[.04em,-.05em]

\newcommand*\Default[1]{%
  \hfill\llap
    {%
      \ifblank{#1}%
        {(initially~empty)}%
        {Default:~\code{#1}}%
    }%
  \newline
}

\usepackage[backend=biber,style=alphabetic]{biblatex}
\addbibresource{\jobname.bib}

% rudimentary solution for a `maintainer' field:
\DeclareFieldFormat{authortype}{\mkbibparens{#1}}
% \DeclareFieldAlias{maintainer}{author}
\DeclareBibliographyAlias{package}{misc}
\renewbibmacro*{author}{%
  \ifboolexpr{
    test \ifuseauthor
    and
    not test {\ifnameundef{author}}
  }
    {\printnames{author}%
     \iffieldundef{authortype}
       {}
       {\setunit{\space}%
	\usebibmacro{authorstrg}}}
    {}}

\usepackage{filecontents}
\begin{filecontents*}{\jobname.bib}
@package{pkg:babel,
  title   = {babel},
  author  = {Johannes Braams},
  date    = {2008-07-08},
  version = {3.8m},
  url     = {http://mirror.ctan.org/macros/latex/required/babel/base}
}
@package{pkg:endnotes,
  title      = {endnotes},
  author     = {Robin Fairbairns},
  authortype = {current maintainer},
  date       = {2003-01-15},
  version    = {NA},
  url        = {http://mirror.ctan.org/macros/latex/contrib/endnotes}
}
@package{pkg:exsheets,
  title   = {exsheets},
  author  = {Clemens Niederberger},
  date    = {2013-04-04},
  version = {0.8a},
  url     = {http://mirror.ctan.org/macros/latex/contrib/exsheets}
}
@package{pkg:fnpct,
  title   = {fnpct},
  author  = {Clemens Niederberger},
  date    = {2013-02-28},
  version = {0.2j},
  url     = {http://mirror.ctan.org/macros/latex/contrib/fnpct}
}
@package{pkg:hyperref,
  title   = {hyperref},
  author  = {Heiko Oberdiek and Sebastian Rahtz},
  date    = {2012-11-06},
  version = {6.83m},
  url     = {http://mirror.ctan.org/macros/latex/contrib/hyperref}
}
@package{pkg:l3kernel,
  title      = {l3kernel},
  author     = {The \LaTeX3 Team},
  date       = {2013-03-14},
  version    = {4469},
  url        = {http://mirror.ctan.org/macros/latex/contrib/l3kernel}
}
@package{pkg:l3packages,
  title      = {l3packages},
  author     = {The \LaTeX3 Team},
  date       = {2013-03-12},
  version    = {4467},
  url        = {http://mirror.ctan.org/macros/latex/contrib/l3packages}
}
@package{cls:memoir,
  title   = {memoir},
  author  = {Peter Wilson and Lars Madsen},
  date    = {2011-03-06},
  version = {3.6j},
  url     = {http://mirror.ctan.org/macros/latex/contrib/memoir}
}
@package{pkg:polyglossia,
  title   = {polyglossia},
  author  = {Arthur Reutenauer and Fran\c cois Charette},
  date    = {2012-04-29},
  version = {1.2.1},
  url     = {http://mirror.ctan.org/macros/xe­tex/la­tex/poly­glos­sia}
}
@package{pkg:scrlfile,
  title   = {scrlfile},
  author  = {Markus Kohm},
  date    = {2012-06-15},
  version = {3.12},
  url     = {http://mirror.ctan.org/macros/latex/contrib/koma-script}
}
@package{pkg:sepfootnotes,
  title   = {sepfootnotes},
  author  = {Eduardo C. Louren\c{c}o de Lima},
  date    = {2013-01-17},
  version = {0.2},
  url     = {http://mirror.ctan.org/macros/latex/contrib/sepfootnotes}
}
@package{pkg:xpatch,
  title   = {xpatch},
  author  = {Enrico Gregorio},
  date    = {2012-10-02},
  version = {0.3},
  url     = {http://mirror.ctan.org/macros/latex/contrib/xpatch}
}
\end{filecontents*}

\usepackage{imakeidx}
\begin{filecontents*}{\jobname.ist}
 heading_prefix "{\\bfseries "
 heading_suffix "\\hfil}\\nopagebreak\n"
 headings_flag  1
 delim_0 "\\dotfill\\hyperpage{"
 delim_1 "\\dotfill\\hyperpage{"
 delim_2 "\\dotfill\\hyperpage{"
 delim_r "}\\textendash\\hyperpage{"
 delim_t "}"
 suffix_2p "\\nohyperpage{\\,f.}"
 suffix_3p "\\nohyperpage{\\,ff.}"
\end{filecontents*}
\indexsetup{othercode=\footnotesize}
\makeindex[options={-s \jobname.ist},intoc,columns=3]

\usepackage{kantlipsum}
\usepackage{etoolbox}
\AtBeginEnvironment{beispiel}{\setfnpct{dont-mess-around}}
\usepackage{enumitem}

\usepackage{marginnote,ragged2e}
\makeatletter
\providecommand*\sinceversion[1]{%
  \@bsphack
  \marginnote{%
    \footnotesize\sffamily\RaggedRight
    \textcolor{black!75}{Introduced in version~#1}}%
  \@esphack}
\providecommand*\changedversion[1]{%
  \@bsphack
  \marginnote{%
    \footnotesize\sffamily\RaggedRight
    \textcolor{black!75}{Changed in version~#1}}%
  \@esphack}
\makeatother

\begin{document}

\section{Licence and Requirements}\secidx{Licence}
Permission is granted to copy, distribute and/or modify this software under the
terms of the \LaTeX\ Project Public License, version 1.3 or later
(\url{http://www.latex-project.org/lppl.txt}). The package has the status
``maintained.''

\enotez needs and loads the following packages: \paket{expl3}~\cite{pkg:l3kernel},
\paket{xparse}, \paket{xtemplate} and \paket{l3keys2e}~\cite{pkg:l3packages},
\paket{xpatch}~\cite{pkg:xpatch} and \paket{scrlfile}~\cite{pkg:scrlfile}.
\secidx*{Licence}

\section{Motivation}\secidx{Motivation}
\enotez is a new implementation of endnotes for \LaTeXe\ since the \paket{endnotes}
package~\cite{pkg:endnotes} has some deficiencies. Nested endnotes, for example, are
not supported, neither is \paket{hyperref}~\cite{pkg:hyperref}. The
\paket{sepfootnotes} package~\cite{pkg:sepfootnotes} also provides means for endnotes
but actually has a different purpose: to separate input and usage both of
footnotes and endnotes. So it might not be the best solution in every
case\footnote{You have to write the actual notes in the preamble or a separate
file and reference them in the text.}. It also does not allow nested endnotes.

While \enotez worked in tests nicely with the \klasse{memoir}~\cite{cls:memoir}
class please keep in mind that \klasse{memoir} provides its own endnote mechanism.

\enotez enables nested endnotes properly and has another mechanism of customizing
the list of endnotes which is easily extendable. One of the main features  of
\enotez is a split list of endnotes in which the notes are automatically
separated by the sections or chapters they were set in, see section~\ref{sec:split}
for more information.

As an aside: \enotez is nicely compatible with the \paket{fnpct} package~\cite{pkg:fnpct}.
Version~0.2j or newer of \paket{fnpct} automatically detects \enotez and adapts
the \cmd{endnote} command.
\secidx*{Motivation}

\section{Usage}\secidx{Usage}
\subsection{Placing the Notes}
The usage is simple: use \cmd{endnote} in the text where you want to place the
note mark.
\begin{beschreibung}
 \Befehl{endnote}[<mark>]{<text>}\newline
   Add an endnote in the text.
\end{beschreibung}
\begin{beispiel}
 This is some text.\endnote{With an endnote.}
\end{beispiel}
There's not really much more to it. It is possible to add a custom mark by
using the optional argument but that shouldn't be needed too often. \cmd{endnote}
works fine inside tables, minipages, floats and captions\footnote{This has been
tested with the standard classes, \klasse{memoir}, and the \klasse{KOMA-Script}
classes, with and without the \paket*{caption} package. If you're using another
package that redefines \cmd*{caption} or are using another class it might not
work. Before you place a note in a caption you should re-think the idea anyway.}.
Endnotes can also be nested.

Since this functionality seemed making a pair \cmd{endnotemark}/\cmd{endnotetext}
superfluous they are \emph{not} defined by \enotez.
\begin{beispiel}
 This is some text.\endnote{With another endnote.\endnote{This is a
 nested\endnote{And another level deeper\ldots} endnote!}}
 % uses package `kantlipsum' to produce dummy text:
 Of course you can have several paragraphs\endnote{\kant[1-3]} in an endnote.
\end{beispiel}

The marks of the endnotes in the running text are printed through the command
\cmd{enotezwritemark} which defaults to \cmd{textsuperscript}. Its argument
contains the current mark which is preceded by \cmd{enmarkstyle}. Both of these
commands can be redefined of course to adapt to custom settings. This can also
be done using options, see section~\ref{sec:options}. The mark of the endnote
that has been set last is stored in \cmd{theendnote} and in \cmd{@currentlabel}.%
\sinceversion{0.6}

This could be used to define a command which writes a mark:
\begin{beispiel}
 \makeatletter
 \def\endnotemark{\@ifnextchar[{\@endnotemark}{\@endnotemark[\theendnote]}}
 \def\@endnotemark[#1]{\enotezwritemark{\enmarkstyle#1}}
 \makeatother
 Text\endnotemark
\end{beispiel}
Please not that this definition \emph{does not} step the endnote counter but
either refers to the last number (no optional argument) or uses the one provided
in the optional argument.

\makeatletter
\def\endnotemark{\@ifnextchar[{\@endnotemark}{\@endnotemark[\theendnote]}}%
\def\@endnotemark[#1]{\enotezwritemark{\enmarkstyle#1}}%
\makeatother

Endnotes can also be labelled and later be referred to:
\begin{beispiel}
 The next endnote\endnote{This endnote gets a label.}\label{en:test}
 has the number~\ref{en:test}. We can use this with the previous definition
 of \string\endnotemark\endnotemark[\ref{en:test}].
\end{beispiel}

\subsection{Printing the Notes}
The notes are printed by using the command \cmd{printendnotes}.
\begin{beschreibung}
 \Befehl{printendnotes}*[<style>]\newline
   Print the list of endnotes. \code{<style>} is one of the instances explained
   in section~\ref{ssec:customizing_the_list}.
\end{beschreibung}
If used without argument it prints all notes set so far with \cmd{endnote}. The
current list will then be cleared. All endnotes set after it are stored again
for the next usage of \cmd{printendnotes}. The starred version will print
\emph{all} endnotes but shouldn't be used more than once if you have nested
endnotes.

It may take several compilation runs until all notes are printed correctly. In
a first run they are written to the \code{aux} file. In the second run they are
available to \cmd{printendnotes}. If you have nested endnotes they will be written
to the \code{aux} file the first time they're printed with \cmd{printendnotes}
which means you might have to compile your file once more. If you change any of
the endnotes or add another one you again will need at least two runs, maybe more.
\enotez tries to warn you in these cases by invoking the warning
\achtung{\code{Endnotes may have changed. Rerun to get them right.}}
but may not catch all cases.

\enotez provides two commands that allow to set some kinds of preamble and
postamble to a list, either to every list or only to the next one:
\begin{beschreibung}
 \Befehl{AtEveryEndnotesList}{<text>}\newline\sinceversion{0.5}%
   inserts \code{<text>} between heading and the actual notes every time
   \cmd{printendnotes} is used.
 \Befehl{AtNextEndnotesList}{<text>}\newline\sinceversion{0.5}%
   inserts \code{<text>} between heading and the actual notes the next time
   \cmd{printendnotes} is used. This overwrites a possible preamble set with
   \cmd{AtEveryEndnotesList} for this instance of \cmd{printendnotes}.
 \Befehl{AfterEveryEndnotesList}{<text>}\newline\sinceversion{0.5}%
   inserts \code{<text>} after the notes list every time \cmd{printendnotes} is
   used.
 \Befehl{AfterNextEndnotesList}{<text>}\newline\sinceversion{0.5}%
   inserts \code{<text>} after the notes list the next time \cmd{printendnotes}
   is used. This overwrites a possible postamble set with
   \cmd{AfterEveryEndnotesList} for this instance of \cmd{printendnotes}.
\end{beschreibung}
If something is inserted with one of these commands the inserted \code{<text>}
will be followed by a \cmd{par} and a vertical skip for the preamble. The postambles
follow a \cmd{par} and a vertical skip. The skips can be set using an option,
see section~\ref{sec:options}.
\secidx*{Usage}

\section{Options}\label{sec:options}\secidx{Options}
\subsection{Package Options}
\enotez has a few package options which should be pretty self-explanatory. They
can be set either as package options with \verb=\usepackage[<options>]{enotez}=
or with the setup command.
\begin{beschreibung}
 \Befehl{setenotez}{<options>}\newline
   Setup command for setting \enotez' options.
 \Option{list-name}{<list name>}\Default{Notes}
   The name of the notes list. This name is used for the heading of the list.
 \Option{reset}{\default{true}|false}\Default{false}
   If set to \code{true} the notes numbers will start from 1 again after
   \cmd{printendnotes} has been invoked.
 \Option{counter-format}{arabic|alph|Alph|roman|Roman}\Default{arabic}
   Change the format of the endnote counter.
 \Option{mark-format}{<code>}\Default{}
   Redefine \cmd{enmarkstyle} to execute \code{<code>}. This command is placed
   directly before the endnote mark in the text.
 \Option{mark-cs}{<command>}\Default{\cmd{textsuperscript}}
   Lets \cmd{enotezwritemark} to be equal to \code{<command>}. This command is
   used to typeset the endnote marks in the text and should take one argument.
 \Option{backref}{\default{true}|false}\Default{false}
   \sinceversion{0.7}If set to \code{true} and \paket{hyperref} has been loaded
   backlinks from the notes in the list to the marks in the text are added.
 \Option{totoc}{section|chapter|false}\Default{false}
   Add an entry to the table of contents.
 \Option{list-heading}{<sectioning command including argument>}\newline
   You can use this option to manually set the list heading command, \textit{e.g.},
   \key{list-heading}{\cmd{chapter}{\#1}} for a numbered heading. The default
   depends upon if the class you're using provides \cmd{chapter} or not. It
   either uses \cmd{chapter}* or \cmd{section}*. You can see that you have to
   refer to the actual heading with \code{\#1}.
 \Option{list-style}{<style>}\Default{plain}
   Sets the default list style, see section \ref{ssec:customizing_the_list} for
   details.
 \Option{list-preamble-skip}{<skip>}\Default{\cmd{medskipamount}}\sinceversion{0.5}%
   Sets the vertical skip (a rubber length) that is inserted if a list preamble
   is inserted by using either \cmd{AtNextEndnotesList} or \cmd{AtEveryEndnotesList}.
   It's default is set equal to \cmd{medskipamount}.
 \Option{list-postamble-skip}{<skip>}\Default{\cmd{medskipamount}}\sinceversion{0.5}%
   Sets the vertical skip (a rubber length) that is inserted if a list postamble
   is inserted by using either \cmd{AfterNextEndnotesList} or \cmd{AfterEveryEndnotesList}.
   It's default is set equal to \cmd{medskipamount}.
\end{beschreibung}

\subsection{Customizing the List}\label{ssec:customizing_the_list}
The list is typeset with \paket{xtemplate}'s possibilities. \enotez declares
the object \code{enotez-list} and two templates for it, the template \code{paragraph}
and the template \code{list}.

\subsubsection{The \code{paragraph} Template}
The \code{paragraph} template's interface is defined as follows:
\begin{beispiel}[code only]
 \DeclareTemplateInterface{enotez-list}{paragraph}{1}
   {
     % parameter   : type       = default
     heading       : function 1 = \section*{#1}   ,
     format        : tokenlist  = \footnotesize   ,
     number        : function 1 = \enmark{#1}     ,
     number-format : tokenlist  = \normalfont     ,
     notes-sep     : length     = .5\baselineskip ,
   }
\end{beispiel}
The parameters functions are these:
\begin{description}[style=nextline]
 \item[\code{heading}] The command with which the heading is typeset.
 \item[\code{format}] The format of the whole list.
 \item[\code{number}] The command that is used to typeset the numbers of the
   notes. The command \cmd{enmark} is explained soon.
 \item[\code{numbers-format}] The format of the numbers.
 \item[\code{notes-sep}] Additional space between the notes.
\end{description}

\enotez uses this template to define the instance \code{plain}:
\begin{beispiel}[code only]
 \DeclareInstance{enotez-list}{plain}{paragraph}{}
\end{beispiel}
This is the default style of the list.

You can easily define your own instances, though:
\begin{beispiel}[code only]
 \DeclareInstance{enotez-list}{custom}{paragraph}
   {
     heading   = \chapter*{#1}        ,
     notes-sep = \baselineskip        ,
     format    = \normalfont          ,
     number    = \textsuperscript{#1}
   }
\end{beispiel}
This would use a chapter heading for the title, separate the notes with
\verb=\baselineskip= and typeset them with \verb=\normalfont=. The numbers would
be typeset with \verb=\textsuperscript=. You could now use it like this:
\begin{beispiel}[code only]
 \printendnotes[custom]
\end{beispiel}

If you wanted superscripted numbers, you could also redefine \cmd{enmark}. 
\begin{beschreibung}
 \Befehl{enmark}\newline
   is defined like this: \verb=\newcommand*\enmark[1]{#1.}=
\end{beschreibung}

\subsubsection{The \code{list} Template}
The \code{list} template's interface is defined as follows:
\begin{beispiel}[code only]
 \DeclareTemplateInterface{enotez-list}{list}{1}
   {
     % parameter   : type       = default
     heading       : function 1 = \section*{#1} ,
     format        : tokenlist  = \footnotesize ,
     number        : function 1 = \enmark{#1}   ,
     number-format : tokenlist  = \normalfont   ,
     list-type     : tokenlist  = description   ,
   }
\end{beispiel}
This template uses a list to typeset the notes. As you can see the default list
is a \code{description} list.

\enotez defines two instances of this template:
\begin{beispiel}[code only]
 \DeclareInstance{enotez-list}{description}{list}{}
 \DeclareInstance{enotez-list}{itemize}{list}{list-type = itemize}
\end{beispiel}
They're available through \cmd{printendnotes}[description] and
\cmd{printendnotes}[itemize], respectively.

Again you can define your own instances using whatever list you want, possibly
one defined with the power of \paket{enumitem}.

\section{Collect Notes Section-wise and Print List Stepwise}\label{sec:split}
\emph{This feature is experimental and surely has some limitations. Please let
me know if something doesn't work as expected}.

Not to be misunderstood: you can use \cmd{printendnotes} as often as you like,
possibly after each section. That is \emph{not} what is meant here. Let's
suppose you are writing a book and have many endnotes in many chapters. It
would be nice if the list of endnotes at the end of the book could be split
into parts for each chapter. This section describes how you can achieve that with
\enotez.

First of all \enotez will rely on the fact that you use \cmd{printendnotes}
only \emph{once}! If you call it more times nobody knows what will happen\ldots

You'll need to tell \enotez that you want to split the notes into groups.
\begin{beschreibung}
 \Option{split}{section|chapter|false}\Default{false}
   Enable the automatic splitting.
 \Option{split-sectioning}{<csname>}\Default{}
   The command that is used to display the titles between the splits. It needs
   to be a command that takes one argument and should be entered without the
   leading backslash. If the option is not used \enotez will choose
   \code{subsection*} for \key*{split}{section} and \code{section*} for
   \key*{split}{chapter}.
 \Option{split-title}{<tokenlist>}\Default{Notes for <name> <ref>}
   The title that will be inserted between the splits. \code{<name>} is replaced
   by \code{section} for \key*{split}{section} and \code{chapter} for
   \key*{split}{chapter}. \code{<ref>} is replaced by the corresponding
   \cmd*{thesection} or \cmd*{thechapter}.
\end{beschreibung}
Set the \key{split} option:
\begin{beispiel}[code only]
 \setenotez{split=section}
\end{beispiel}
Well -- that's it, basically. You'll have to be careful, though:
If you're having nested endnotes the nested ones appear first in the ``Notes''
section (or chapter, respectively). In this case you should have a numbered
section title for the notes, presumably in the appendix. You'll need to create
a new list style:
\begin{beispiel}[code only]
 % preamble:
 \usepackage{enotez}
 \DeclareInstance{enotez-list}{section}{paragraph}{heading=\section{#1}}
 \setenotez{list-style=section,split=section}
 % document:
 \appendix
 \printendnotes
\end{beispiel}

Please beware that the option \key{reset} also impacts here: the numbing will
be reset for each section or chapter, depending on the choice you made for
\key{split}.

\sinceversion{0.7}There are two -- or three, actually -- additional commands:
\begin{beschreibung}
 \Befehl{AtEveryListSplit}{<code>}\newline
   Inserts \code{<code>} before each sub-heading in a splitted list.
 \Befehl{AfterEveryListSplit}{<code>}\newline
   Inserts \code{<code>} after each sub-heading in a splitted list.
 \Befehl{EnotezCurrentSplitTitle}\newline
   Holds the current sub-heading in a splitted list and may be used in
   \cmd{AtEveryListSplit} and \cmd{AfterEveryListSplit}.
\end{beschreibung}


\enotez comes with an example document for a split list which you should
find in the same folder as this documentation.
\secidx*{Options}

\section{Language Support}\secidx{Language Support}
\enotez uses the \paket*{translations} package (part of the
\paket{exsheets}~\cite{pkg:exsheets} bundle) to translate language dependent
strings. The advantage of this is that language settings with
\paket{babel}~\cite{pkg:babel} or \paket{polyglossia}~\cite{pkg:polyglossia}
are detected automatically. However, the available translations are somewhat
limited due to my limited language knowledge. If you find missing or wrong
translations you can try to add or correct them by adding the corresponding
versions of the following lines to your preamble:
\begin{beispiel}[code only]
 \DeclareTranslation{English}{enotez-title}{Notes}
 % ``<name> <ref>'' is a placeholder for e.g. ``section 1'':
 \DeclareTranslation{English}{enotez-splitted-title}{Notes for <name> <ref>}
 \DeclareTranslation{English}{enotez-section}{section}
 \DeclareTranslation{English}{enotez-chapter}{chapter}
\end{beispiel}
If you like you can also drop me an email at
\href{mailto:contact@mychemistry.eu}{contact@mychemistry.eu} and I'll add the
correct translations to \enotez.
\secidx*{Language Support}

\section{hyperref Support}\secidx{\paket*{hyperref} Support}[hyperref Support]
If \paket{hyperref} is loaded and you are using the option \key{totoc} (see
p~\pageref{key:totoc}) the list title is linked via a \verb=\phantomsection=.

If \paket{hyperref} is used with \code{hyperfootnotes} set to \code{true} the
endnote marks are linked to the respective entries in the list\changedversion{0.7}.
If you also set \enotez' option \key{backref} (see p~\pageref{key:backref}) the
notes in the list are themselves linked to the marks in the text.
\secidx*{\paket*{hyperref} Support}[hyperref Support]

\AtNextEndnotesList{This is an example of a preamble to the list set with
\cmd{AtNextEndnotesList}.}
\AfterEveryEndnotesList{\noindent This is an example of a postamble to the list
set with \cmd{AfterEveryEndnotesList}. Note that it would have started with a
paragraph indent which was prevented here by using \cmd{noindent}.}
\printendnotes[addsec]

\printbibliography

\indexprologue{\noindent Section titles are indicated \textbf{bold}, packages
\textsf{sans serif}, commands \code{\textbackslash\textcolor{code}{brown}}
 and options \textcolor{key}{\code{yellow}}.\par\bigskip}

\printindex
\end{document}