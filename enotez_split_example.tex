% --------------------------------------------------------------------------
% the ENOTEZ package
% 
%   Endnotes for LaTeX2e
% 
% --------------------------------------------------------------------------
% Clemens Niederberger
% Web:    https://github.com/cgnieder/enotez/
% E-Mail: contact@mychemistry.eu
% --------------------------------------------------------------------------
% Copyright 2012--2020 Clemens Niederberger
% 
% This work may be distributed and/or modified under the
% conditions of the LaTeX Project Public License, either version 1.3
% of this license or (at your option) any later version.
% The latest version of this license is in
%   http://www.latex-project.org/lppl.txt
% and version 1.3 or later is part of all distributions of LaTeX
% version 2005/12/01 or later.
% 
% This work has the LPPL maintenance status `maintained'.
% 
% The Current Maintainer of this work is Clemens Niederberger.
% --------------------------------------------------------------------------
% The enotez package consists of the files
%  - enotez.sty, enotez_en.tex, enotez_en.pdf, README
% --------------------------------------------------------------------------
% If you have any ideas, questions, suggestions or bugs to report, please
% feel free to contact me.
% --------------------------------------------------------------------------

\documentclass{article}
% \documentclass{scrartcl}
% \documentclass{memoir}
\usepackage[english]{babel}
% \usepackage{caption}
\usepackage[T1]{fontenc}
\usepackage{libertine}

\usepackage{enotez}
\usepackage{kantlipsum}

\DeclareInstance{enotez-list}{itemize}{list}{list-type=itemize,heading=\section{#1}}
\setenotez{
  split=section,
  list-style=section,
  % reset,
  backref
% customize the titles in between, e.g.:
%   split-sectioning=addsec,
%   split-title=\par\noindent<name> <ref>:
}

\usepackage{fnpct}

\usepackage{mwe}
\usepackage[colorlinks]{hyperref}

\begin{document}

\tableofcontents

\section{Test}
Text\endnote{\kant[1]}. Text\endnote{\kant[2]}. Text\endnote{This time
with a \textbf{nested}\multendnote{\kant[3];\kant[4]} endnote.}.
Text\multendnote{\kant[5];\kant[6]}.

\begin{figure}[htp]
  \centering
  \includegraphics[width=.4\linewidth]{example-image-a}
  % \addtocounter{endnote}{-1} % <<< need this in memoir class and with
                               % `caption' package when using an endnote
                               % inside a caption
  \caption[caption]{Some text\endnotemark.}
  \endnotetext{An endnote with a nested endnote\endnote{\kant[7]} inside a
    figure caption.}
\end{figure}

\appendix
\printendnotes[itemize]

\end{document}

\tableofcontents

\section{Test}
Text\endnote{\kant[1]}. Text\endnote{\kant[2]}. Text\endnote{This time
with a \textbf{nested}\multendnote{\kant[3];\kant[4]} endnote.}.
Text\multendnote{\kant[5];\kant[6]}.
\begin{figure}[htp]
 \centering
 \includegraphics[width=.4\linewidth]{example-image-a}
 \caption[caption]{Some text\endnote{An endnote with a nested endnote\endnote{\kant[7]}
 inside a figure caption.}.}
\end{figure}

\section{Test}
Text\endnote{\kant[8]}. Text\endnote{\kant[9]}. Text\endnote{This time
with a \textbf{nested}\endnote{\kant[10]} endnote.}. Text\endnote{\kant[11]}.

\appendix
\printendnotes

\end{document}
